% This is samplepaper.tex, a sample chapter demonstrating the
% LLNCS macro package for Springer Computer Science proceedings;
% Version 2.20 of 2017/10/04
%
\documentclass[runningheads]{llncs}
%
\usepackage{graphicx}
% Used for displaying a sample figure. If possible, figure files should
% be included in EPS format.
%
% If you use the hyperref package, please uncomment the following line
% to display URLs in blue roman font according to Springer's eBook style:
% \renewcommand\UrlFont{\color{blue}\rmfamily}
\begin{document}
%
\title{Delay Analysis for URLLC in 5G Based on Stochastic Network Calculus\thanks{Supported by program National Natural Science Foundation of China. (Nos.61370065,61502040).
Beijing Municipal Program for Excellent Teacher Promotion. (No.PXM2017\_014224.000028).}}
%
%\titlerunning{Abbreviated paper title}
% If the paper title is too long for the running head, you can set
% an abbreviated paper title here
%
\author{Shengcheng Ma\inst{1}\orcidID{0000-0003-1060-1208} \and
Zhuo Li\inst{1}\orcidID{1111-2222-3333-4444} \and
Xin Chen\inst{1}\orcidID{2222--3333-4444-5555}}
%
\authorrunning{SC Ma et al.}
% First names are abbreviated in the running head.
% If there are more than two authors, 'et al.' is used.
%
\institute{Beijing Information Science and Technology University, Beijing 08544, China \and
Springer Heidelberg, Tiergartenstr. 17, 69121 Heidelberg, Germany
\email{mashengcheng@163.com}\\
\url{http://www.springer.com/gp/computer-science/lncs} \and
ABC Institute, Rupert-Karls-University Heidelberg, Heidelberg, Germany\\
\email{\{abc,lncs\}@uni-heidelberg.de}}
%
\maketitle              % typeset the header of the contribution
%
\begin{abstract}
The fifth generation (5G) wireless networks are upcoming to our life.
The higher performance requirements are raised to satisfy the needs in modern communication.
Ultra-reliable low latency communications (URLLC) is one of the most important scenarios in 5G.
URLLC with strict latency and reliability requirements is widely used in some delay-sensitive applications such as self-driving.
As the 3GPP claimed, the URLLC is amenable to 99.999\% transmission correctness and within 1ms delay bound.
How to meet the requirements of reliability and latency is still an open issue.
Some academic studies and companies proposed various methods to design URLLC standard, but little effort has been made on applying a theoretical method to analyze the delay bound.
Stochastic network calculus is an elegant way to obtain the delay bound based on traffic models and service guarantees.
In this paper, we take the character of 5G architecture into account and use the stochastic network calculus to analyze the delay in URLLC.
Some factors which can influence on the delay are obtained.
Optimizing these factors to reduce the delay will provide valuable guidelines for the early design of URLLC architecture.
Finally, numerical results are presented to verify the correctness of the delay analysis.

\keywords{5G \and URLLC \and Stochastic Network Calculus \and Delay Analysis.}
\end{abstract}
%
%
%
\section{Introduction}
The 5G era is getting closer to us.
5G communication technology appeared for the first time with the 2018 Pyeongchang Winter Olympics in South Korea.
It helps audiences watch the live broadcast continuously and smoothly.
According to International Telecommunication Union announced the 5G standard timetable, 5G will start commercially in 2020 \cite{ref_standard1}.
5G wireless networks are designed to support diverse and complicated scenarios.
The third generation partnership project (3GPP) classify these different scenarios into three big categories: enhanced mobile broadband (eMBB), massive machine type communications (mMTC), and ultra-reliable low-latency communications (URLLC) \cite{ref_standard2}.

URLLC is widely used in self-driving, mission critical application and some delay sensitive systems.
It has stringent requirements in terms of delay and reliability in the 5G New Radio (NR) systems.
The key requirements of URLLC as claimed by the 3GPP are to ensure the latency of user plane data less than 1ms for downlink and uplink, meanwhile to keep very high packet reception reliability about 99.999 percent.\cite{ref_article1}
The stringent delay requirement needs new 5G NR technology to bridge the gap.
Though the existing LTE networks can reach the reliability target, but the cost is some dozens of milliseconds time delay.
That is far away from the criteria of URLLC.
So the delay becomes the chock point and it needs to be solved.
Many academies and companies have proposed some engineering solutions to minimize the delay.
Such as the HARQ retransmission or grant-free technology.
However, how to analyze the generation of time-delay from a theoretical perspective and propose a strategy to reduce the delay effectively is an important research subject.

Stochastic network calculus (SNC) theory is very good at delay performance analysis.
The SNC is a continuous development method to analyze network traffic characteristic and evaluate performance\cite{ref_book1}.
Different from queuing theory, the SNC permits some packets violate the desired performance.
This feature can better take advantage of statistical multiplexing gains\cite{ref_article2}.
To deal with random service and statistical guarantee, the SNC theory comes into being with a large number of stochastic processes and network traffic models.
Under a suitable traffic model and a chosen server model, the SNC theory can process service guarantee analysis of communication network like delay and backlog.
So we capitalize on the SNC method to analyze the delay of the 5G URLLC transmission in this paper.

We use stochastic arrival curve to describe the process that user equipment (UE) data sends to gNodeB (gNB) side.
According to the 5G network topology architecture, we can deduce the rest stages of data transmission from gNB to cloud server.
Every stage of stochastic arrival curve characterizes the delay property, therefore the whole delay of URLLC system is comprised by delays which generated from UE to cloud server.

Our main contributions of this paper can be summarized as follows:

1) We build a tandem model to simulate 5G network architecture. In this model, we can analog the data transmission in uplink or downlink from UE to cloud server. We use stochastic service process and concatenation property to analysis the latency.

2) Our analysis results represent which parameters are the key factors affecting the delay. By adjusting the key factors, we give a group of tactics to reduced the delay effectively.

3) Delay analysis and tactics for reducing latency have valuable theoretical guidance for the design of URLLC deployment. In order to meet stringent delay requirements, it provides guidelines for how to allocate resources.

The rest of this paper is organized as follows.
Section II summarizes related work of URLLC technology and stochastic network calculus.
We present a tandem network model to describe URLLC in Section III.
In particular, we illustrate the architecture of this system and analyze the causes of the delay in this section.
In section IV, we introduce the experimental environment and analyze the relationship between latency and main factors.
We conclude this paper in Section V.
Some theoretical proofs are given in appendix.

\section{Related Work}
Because the standard of URLLC has not been worked out, many researchers have put forward different solutions for the design of URLLC.

Dozens of researches are focus on how to design and implement URLLC to meet the performance requirements.
A design without intervention in the baseband/PHY layer for URLLC is to use interface diversity and integrate multiple communication interfaces.
Jimmy and his colleagues propose an analysis framework that combines traditional reliability models with technology-specific latency probability distributions\cite{article_interface}. 
In this way, they can estimate the performance in terms of latency and reliability in such an integrated communication system.
To guarantee a low end-to-end delay with low jitter over combined internet and wireless interfaces, the article \cite{article_multiconnectivity} presents a new multiple-input multiple-output(MIMO) networked round trip time (RTT) skew control controller.
This controller's advantage is that it solves the data flow split problem at the controlling node. 
Jaya Rao and Sophie Vrzic have propose an approach to adopt packet duplication (PD) method to satisfy the latency and reliability requirements\cite{article_PD}. 
PD technology generates multiple instances and sends them simultaneously in multiple unrelated channels. 
The receiver selects the best packet according to the channel condition in order to achieve better transmission reliability. 
This PD technique can provide a cost-effective solution without increasing the complexity in the radio access network (RAN).

In terms of resource allocation and energy efficiency, there are also some researches on URLLC.
How frequency resource are allocated to send a user data in URLLC scenario.
That is an interesting study which plunged by Anand A and De Veciana G\cite{article_Anand}.
Based on the 5G standard technology Orthogonal Frequency Division Multiple Access (OFDMA), they build a One Shot Transmission model which does not allow retransmission.
Adopting queuing theory analysis, they find out a result that a small bandwidth over a longer duration is better than a large swath of bandwidth for short duration in One Shot Transmission system.
Green energy saving is getting more and more attention.
The article \cite{article_Energy} provides a coordinated on-off switching scheme across a set of adjacent gNBs.
The gNBs share a sleep schedule among themselves. 
It will select an OFF durations if gNBs have lower traffic and fewer connected UEs.
This on-off mode is more energy-efficient than traditional mode on the premise of guaranteeing the time delay.

Because URLLC has strict requirements for delay and reliability, it is very meaningful to evaluate the performance of URLLC.
Joachim et al. provide an achievable latency bound evaluation in their article\cite{joachim}.
They compare the worst case RAN transmission latencies for different 5G URLLC configurations.
The configuration contains FDD, TDD, frequency numerologies and usage of slots.
According the analysis, a frequency with higher numerology can be used to reduce the latency.
An article derived from HUAWEI propose a grant-free mode uplink transmission mechanism\cite{Huawei}.
Grant-free transmission is a transmission without scheduling request and dynamic grant.
This scheme is poised to meet the reliability requirement of URLLC in uplink transmission.
By simulating different numbers of active UEs random arriving, the reliability can be improved after adopting the grant-free mode with increasing retransmission.

\subsection{A Subsection Sample}
Please note that the first paragraph of a section or subsection is
not indented. The first paragraph that follows a table, figure,
equation etc. does not need an indent, either.

Subsequent paragraphs, however, are indented.

\subsubsection{Sample Heading (Third Level)} Only two levels of
headings should be numbered. Lower level headings remain unnumbered;
they are formatted as run-in headings.

\paragraph{Sample Heading (Fourth Level)}
The contribution should contain no more than four levels of
headings. Table~\ref{tab1} gives a summary of all heading levels.

\begin{table}
\caption{Table captions should be placed above the
tables.}\label{tab1}
\begin{tabular}{|l|l|l|}
\hline
Heading level &  Example & Font size and style\\
\hline
Title (centered) &  {\Large\bfseries Lecture Notes} & 14 point, bold\\
1st-level heading &  {\large\bfseries 1 Introduction} & 12 point, bold\\
2nd-level heading & {\bfseries 2.1 Printing Area} & 10 point, bold\\
3rd-level heading & {\bfseries Run-in Heading in Bold.} Text follows & 10 point, bold\\
4th-level heading & {\itshape Lowest Level Heading.} Text follows & 10 point, italic\\
\hline
\end{tabular}
\end{table}


\noindent Displayed equations are centered and set on a separate
line.
\begin{equation}
x + y = z
\end{equation}
Please try to avoid rasterized images for line-art diagrams and
schemas. Whenever possible, use vector graphics instead (see
Fig.~\ref{fig1}).

\begin{figure}
\includegraphics[width=\textwidth]{fig1.jpg}
\caption{A figure caption is always placed below the illustration.
Please note that short captions are centered, while long ones are
justified by the macro package automatically.} \label{fig1}
\end{figure}

\begin{theorem}
This is a sample theorem. The run-in heading is set in bold, while
the following text appears in italics. Definitions, lemmas,
propositions, and corollaries are styled the same way.
\end{theorem}
%
% the environments 'definition', 'lemma', 'proposition', 'corollary',
% 'remark', and 'example' are defined in the LLNCS documentclass as well.
%
\begin{proof}
Proofs, examples, and remarks have the initial word in italics,
while the following text appears in normal font.
\end{proof}
For citations of references, we prefer the use of square brackets
and consecutive numbers. Citations using labels or the author/year
convention are also acceptable. The following bibliography provides
a sample reference list with entries for journal
articles~\cite{ref_article1}, an LNCS chapter~\cite{ref_lncs1}, a
book~\cite{ref_book1}, proceedings without editors~\cite{ref_proc1},
and a homepage~\cite{ref_url1}. Multiple citations are grouped
\cite{ref_article1,ref_lncs1,ref_book1},
\cite{ref_article1,ref_book1,ref_proc1,ref_url1}.
%
% ---- Bibliography ----
%
% BibTeX users should specify bibliography style 'splncs04'.
% References will then be sorted and formatted in the correct style.
%
% \bibliographystyle{splncs04}
% \bibliography{mybibliography}
%
\begin{thebibliography}{8}
\bibitem{standard1}
ITU-R M.2083-0.: IMT Vision - Framework and Overall Objectives of the Future Development of IMT for 2020 and Beyond. (2015).
\bibitem{standard2}
 3GPP TR 38.913.: Study on Scenarios and Requirements for Next Generation Access Technologies. (2017).
\bibitem{article1}
Soldani D, Guo Y J, Barani B, et al.: 5G for Ultra-Reliable Low-Latency Communications. IEEE Network \textbf{32}(2), 6--7 (2018)
\bibitem{book1}
Jiang Y, Liu Y.: Stochastic Network Calculus. Springer, London, (2009)
\bibitem{article2} 
M. Fidler and A. Rizk.: A Guide to the Stochastic Network Calculus. IEEE Communications Surveys \& Tutorials, \textbf{17}(1), 92--105 (92-105)
\bibitem{article_interface} 
J. J. Nielsen, R. Liu and P. Popovski.: Ultra-Reliable Low Latency Communication Using Interface Diversity. IEEE Transactions on Communications \textbf{66}(3), 1322-1334 (2018)
\bibitem{article_multiconnectivity} 
Delgado R A, Lau K, Middleton R H, et al.: Networked Delay Control for 5G Wireless Machine-Type Communications Using Multiconnectivity. IEEE Transactions on Control Systems Technology \textbf{99}, 1--16 (2018)
\bibitem{article_PD} 
Rao J, Vrzic S.: Packet Duplication for URLLC in 5G: Architectural Enhancements and Performance Analysis. IEEE Network textbf{32}(2) 32--40 (2018)
\bibitem{article_Anand} 
Anand A, De Veciana G.: Resource Allocation and HARQ Optimization for URLLC Traffic in 5G Wireless Networks. (2018)
\bibitem{article_Energy} 
Mukherjee A.: Energy Efficiency and Delay in 5G Ultra-Reliable Low-Latency Communications System Architectures. IEEE Network textbf{32}(2) 55--61 (2018)


\bibitem{ref_article_1}
Author, F.: Article title. Journal \textbf{2}(5), 99--110 (2016)
\bibitem{ref_lncs1}
Author, F., Author, S.: Title of a proceedings paper. In: Editor,
F., Editor, S. (eds.) CONFERENCE 2016, LNCS, vol. 9999, pp. 1--13.
Springer, Heidelberg (2016). \doi{10.10007/1234567890}

\bibitem{ref_book_1}
Author, F., Author, S., Author, T.: Book title. 2nd edn. Publisher,
Location (1999)

\bibitem{ref_proc1}
Author, A.-B.: Contribution title. In: 9th International Proceedings
on Proceedings, pp. 1--2. Publisher, Location (2010)

\bibitem{ref_url1}
LNCS Homepage, \url{http://www.springer.com/lncs}. Last accessed 4
Oct 2017
\end{thebibliography}
\end{document}
